
\section{Introduction}

%\enlargethispage{30pt}

The technological evolution of sequencing platforms has progressed rapidly since the completion of the Human Genome project via Sanger sequencing methods \citep{lander2001initial,sanger1977dna}. Modern high-throughput sequencing (HTS) approaches post-Sanger era have superseded this standard, allowing for a greater number of variants to be sequenced across the whole genome by employing powerful mass fragmentation/amplification approaches upon a target sequence \citep{lengauer2007bioinformatics,pabinger2014survey}.

The raw sequence FASTQ reads produced by these HTS platforms are aligned to a specific version of the NCBI reference sequence and collated into a Binary Alignment Map (BAM) where variants of interest can then be individually "called" to form a Variant Call Format (VCF) file of novel or known variants conforming to a specific variant database (dbSNP) \citep{li2009sequence,danecek2011variant}.

BAM and VCF data are orthogonally related, with the former storing horizontal stretches of FASTA sequence reads aligned unevenly on top of one another forming "pile ups", and the latter taking vertical cross-sections of these pileups at specific loci to form a variant call.

%; e.g.  a vertical-section of a pileup of interest numbering $n$ reads of one nucleobase and $n$ reads of a different nucleobase can be reported as a high quality heterozygous variant with a total read depth of $2n \text{ for large } n$.

%\fig{fig:pileup}{images/keep/pileup.jpg}
%{Visualization of a BAM file using Golden Helix Genome Browser. (a) Read depth representing the amount of reads that have been aligned to the reference sequence (c) for each bp. The pile-up (b) shows each read in the given region. The red column indicates a single variant, with summarized variant properties shown in table (d).}

\fig{fig:vcf}{images/keep/vcf_header.jpg}
{VCF header and body data describing the type of annotations and their contents respectively.}


The VCF specification was designed for the 1000 Genomes project to produce a robust format that could house the many samples often sequenced under the same batch, but has since been adopted by other projects such as UK10K, dbSNP, NHLBI Exome Project, and others. As shown in Fig~\ref{fig:vcf}, the format is flexible with annotations, where additional fields can be outlined in the header and adhered to in the body of the data \citep{danecek2011variant}. 

Each line of the VCF body describes a single variant; physical position paired with a reference allele (as ascribed by a reference genome consistent across the entire VCF file) and alternate alleles that appear within samples. Major and minor alleles are specific only to the sample population but their frequencies can be pre-computed and appended to a variant line as additional information to then be utilized in small population analyses such as inheritance modelling \citep{danecek2011variant}.

Variant analysis suites all work under the same principle; filtering all variants under a user-specified set of criteria against the various variant annotations present in the VCF in order to produce a subset informative to the phenotype. Optimistic filtering measures will produce a smaller set with the drawback of missing key causative variants, and conservative filtering measures will produce too many false positives.

The effectiveness of an analysis rests primarily upon the accuracy of the variant annotations which can attribute to as much as 15\% of false negatives \citep{warden2014detailed}, as well as the frequency of false negatives that are discarded due to overly-stringent quality filtering. A common approach to addressing both issues is through learning algorithms that can be trained to favour individual variants over others with the caveat of producing results via 'black-box' methods that may create some disparity between the user and their data \citep{pabinger2014survey}. 

A more transparent approach is to expand the scope of the filtering beyond the variant/gene-level and explore variants under a larger trait-penetrance context outlined in Fig~\ref{fig:inheritance}. 

Mendelian traits conform to the four classical modes on inheritance of autosomal/X-linked dominant/recessive penetrance. Dominant disorders result from the inheritance of a single mutant allele which is manifested in each subsequent generation with a 50\% chance of likelihood in offspring from a single affected parent. Autosomal and X-linked dominant models are identical with the exception of the transmitting chromosome.

Recessive traits require the inheritance of two mutant alleles on opposing strands in order to completely block any functioning copies of the causative gene. Parents are typically carriers with affected offspring. These disorders are primarily a result of consanguineous marriages, where a single mutant allele manifests on both alleles due to the multiple paths of descent it can undertake \cite{lander2001initial}. In the case of X-linked recessive inheritance, males with a single mutant copy are hemizygous and are forced to express the phenotype.

For non-Mendelian disorders, we also consider the special case of \textit{mosaicism}; where embryonic de novo mutations produce two or more populations of cells that result in segregated sets of genotypes within the same individual. Mosaic genotypes can be revealed stochastically by measuring alternate allele frequencies against expected values \citep{biesecker2013genomic}.

Here we outline an open-source variant analysis suite that makes use of these inheritance modelling scenarios with the aim to vastly reduce the number of false positives.


\fig{fig:inheritance}{images/keep/inh_autosomal.jpg}
	{Autosomal inheritance pattern with red disease allele, both parents as 
(A) Recessive inheritance, both parents as heterozygous carriers with an unaffected:carrier:affected ratio of 1:2:1.
(B) Dominant inheritance, one parent affected with an unaffected:affected ratio of 1:1.}
	
\fig{fig:inheritance}{images/keep/inh_xlinked.jpg}
	{X-linked inheritance pattern with a red disease allele.
(A) Recessive inheritance, mother is a disease-carrier. A mother will pass her disease-allele
to half of her offspring, resulting just sons to be affected.
(B) Recessive inheritance, father is affected. A father will pass his disease-allele to all of
his daughters resulting all of them to be a disease-carriers.
(C) Dominant inheritance, father is affected. A father will pass his disease-allele to all his
daughters resulting all of them to be affected. 
(D) Dominant inheritance, mother is affected. A mother will pass her disease-allele
equally to daughters and sons, resulting half of them to be affected.}

