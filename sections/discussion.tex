
%% Results table

\begin{table}[!t]
\processtable{Average single-core runtimes of VCF files containing 50,000 variants passing individually through all filters\label{table:results}}
{\begin{tabular}{| c | *2c |} \hline %\toprule
%\emph{Pipeline} & \emph{Module} & \emph{Runtime} \\
%\emph{Stage} & \emph{Name} & \emph{(seconds)} \\
\emph{Pipeline Stage} & \emph{Module Name} & \emph{Runtime (seconds)} \\
\hline
% A multirow would be more benefial here
% but let's keep package req down for now
             & Adding Genes     & 125\\
Annotation   & Adding Function & 28.7\\
             & Adding Zygosity & 0.81\\
\hline
             & Physical Location Filter & 1.02 \\
             & Read Depth Filter        & 1.26 \\
             & Call Quality Filter      & 0.93\\
Filtering    & AAF Filter               & 432\\
             & Mutation Type Filter     & 1.08\\
             & Novel Variant Filter     & 1.12\\
             & Same Gene Filter       & 22.5\\
             & Same Variant Filter    & 26.1\\
\hline
             & AD Inheritance    & 0.83\\
 Trait       & AR Inheritance    & 1.22\\
Penetrance   & XD Inheritance    & 0.74\\
  Model      & XR Inheritance    & 1.39\\
             & Mosaicism         & 0.94\\
\hline
                 & Isoform Context      & 2.28\\
   Extended      & Protein Context      & 4.10\\
 Annotation      & Gene Expression      & 145\\
%                 & Transcription Factor & 0.86\\
%                 & House-Keeping        & 0.93\\
\hline
%\botrule
\end{tabular}}
{Single VCF file timings for Annotation,Filtering, and Extended Annotation modules. Trait Penetrance module timings are based on three VCFs consisting of a parent-offspring trio. Tests were run on a 2GHz dual-core processor with 4GB RAM.}
\end{table}


\section{Discussion}

Depending upon the total input variants as well as the number and ordering of modules used, an average initial analysis using any number of modules (excluding alternate allele filtering) for VCF files containing 300,000 variants each, will attribute a total of 2 minutes per VCF.

There are several limiting steps however, with the largest bottleneck occurring at initial gene annotation stage, which must prime all input variants for downstream filtering through the use of a gene (or exon) map that is dependent upon user parameters. Gene maps for a variety of user parameters already exist as static files in the live environment, but not all use-cases are covered and a new gene map must be generated for custom configurations which can take up to 1 hour to retrieve depending on internet speed and proximity to the closest UCSC MySQL mirror.

In the case of general gene map use-cases, the \textit{Adding Genes} annotation step still requires 200 times more processing time than most other modules, and was the sole reason that all annotation modules were re-written in C++ to benefit from a significant performance increase that reduced the module's processing time from an initial time of 10 minutes to under 3 minutes (Table~\ref{table:results}). 

The rest of the annotation modules are comparatively much faster, with the functional annotations experiencing mild latency related to disk read speeds when performing repeated byte-offset lookup upon FASTA files. The initial sorting of the variants upon file upload is valuable in this regard due to the higher tendency of adjacent variants to share the same disk cluster and reap paging benefits.

The last noticeable slowdown occurs within the Javascript-powered interactive report and is dependent upon the number of final variants it has to tabulate, where the difference between 1,000 and 10,000 final variants maps to a range of 1 to 15 seconds.

Across subsequent pipeline runs, processing is not repeated for the same data; each module checks whether an input VCF file has already been processed by the current pipeline configuration, and repeatedly iterates through the module ordering until the last processed input set is reached where it can resume processing.




\subsection{Transparency and Deployment}

The portability of \app grants a significant advantage over present-day web-based pipelines by keeping all analyses securely \textit{in situ}, which is greatly beneficial to regions of the world without consistent or active internet in addition to researchers handling personal or private data.

Cloud-based analyses require input data to be uploaded to an external server in order to perform processing, and data ownership after upload is not always retained especially in the case where the work was performed within the cloud \citep{reed2010information}.

Further, many cloud-services employ non-transparent proprietary methods to reduce the number of false-positives and false-negatives. A common approach is to make use of an internal database or learning algorithm that favours some variants over others based on previous analyses (or a similar training set) \citep{pabinger2014survey}, resulting in informative variants produced by unquantifiable "black-box" means, creating disparity between the end-user and their analysis.

Transparent filtering methods are likelier to instil greater confidence in the data with the added benefit of customization to better tailor a filter to an analysis in the case of open-source implementations, as with the case of \app.
