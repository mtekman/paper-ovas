
\section{Conclusion}

The self-contained environment provided by \app allows researchers to tailor all aspects of their analysis and retain control of their data sets at any phase of processing by means of the transparent open-source modules that comprise the pipeline. 

The live environment, paired with the web front-end, provides the additional advantage of abstracting the end-user from the underlying platform specifics by streamlining the input and configuration process, as well as logging active progress descriptions for the current stage of processing, and lastly providing a malleable final report upon all remaining variants discovered complete with dynamic filtering capabilities.

The entirety of all uploaded variants are processed first at the gene annotation stage, placing significant strain at the initial stage of the pipeline that is only managed through the use of employing C++ binaries to overcome the performance bottleneck that would otherwise exist with Python/Bash scripts.

The annotation step is crucial however, especially for whole-genome sequence data where the vast majority of the variants would be deemed wholly intergenic and would be filtered out as uninformative to the analysis. More common exome-sequencing data typically observe less of a reduction at a much faster processing rate due to the smaller number of total variants, but at the impediment of missing regulatory elements due to lack of coverage.

Modules downstream of the annotation stage run trivially, and due to the pipeline's resume feature which prevents \app from processing the same data twice, many subsequent analyses with different module configurations can be run in quick succession after the initial annotation step is complete.

\app is future-secure due to the inclusion of the background scripts that generated the static data being packaged with the live environment. Updates to the human genome reference, variant databases, and FASTA sequences can be retrieved on demand for platforms with active internet connections. Changes will preserve across successive boots for non-volatile storage mediums such as USB sticks, ideal in deployment scenarios with infrequent or absent internet access.

%A number of future improvements are to be expected, with further streamlining of the file input process through the web front-end by means of integrating a pedigree creation tool in-browser that will generate a pedigree file from families drawn by the user within a HTML5 canvas.

%Planned enhancements to the core \app back-end will incorporate parallelization to the over-arching framework in order to process independent files or modules asynchronously, which as shown in Table~\ref{table:results} would lend no significant performance increase in current filtering modules due to their trivial runtimes, but will become more essential in the future for inevitably denser VCF input sets.

%More immediate changes will focus on implementing additional feature annotation and filtering modules such as alternative splice sites and transcription factor binding sites, as well as extending existing modules parameters to incorporate special-case scenarios such as mitochondrial penetrance and cancer-specific mosaic filtering.

