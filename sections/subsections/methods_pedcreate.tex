
\subsection{Pedigree Creation}

In order to be a fully comprehensive pedigree suite, creation tools are available to address the more clinical aspects of the application.

Pedigrees are compliant with the Pedigree Standardization Work Group (PSWG) specification \citep{pswg1,pswg2} familiar to most clinicians, and provides a ready means to record sensitive patient data through member-specific annotations. 

Creation utilities are localised to two intuitive modes of drawing relationships; where lines are drawn from one individual to another in order to indicate heterosexual partnership via \textit{Matelines}, or from one individual to a Mateline in order to indicate a parent-offspring trio via \textit{Childlines}. Lines snap to context-dependent anchor points that become visible as soon as the drawing of a line is initialised as shown in Fig~\ref{fig:pedcreate}. Connected individuals anchored on the same line are vertically aligned to one another such that both members of a Mateline move as a single unit, as well as siblings bound to the same line via their respective Childlines.

Projects are not limited to single families, and complex consanguineous relationships are automatically detected and represented via double-lines. Pedigrees can be imported/exported via the standard LINKAGE (pre-makeped) format, or saved and resumed from local browser storage for ongoing projects.

\placeholderimage{fig:pedcreate}
	{PedCreate View displaying the four stages of creating a pedigree: (1) Adding individuals and modifying their properties, (2) Joining mates through a Mateline with anchor points made visible within red circles, (3) Joining offspring to their parents through a Childline with anchor points made visible with white circles, (4) Completing a trio.}


% Pre-makeped
% http://www.helsinki.fi/~tsjuntun/autogscan/pedigreefile.html