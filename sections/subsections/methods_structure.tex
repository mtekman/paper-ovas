
\subsection{Pipeline Modules}

Each module is tasked with the function of separating variants from an input file into two distinct output VCF groups of "filtered" and "discarded"; with the former group being passed into the next module, and the latter being halted at the current point of processing to be stored for potential debugging purposes.

The discard process at each module lends a progressive performance increase in the processing speed of each subsequent module due to the input being only a subset of the input that came before it, whilst still retaining the aggregate total of discarded variants at each step.

\subsubsection{Pre-processing}

All VCF files immediately undergo initial preparation upon file upload from the web interface, where a background shell script renames the files to better emulate their pedigree counterparts, and asserts that all variants are in correct order following a chromosome:position sorting key.


\subsubsection{Core Annotation}

The annotation stages of the pipeline then prime the variants with relevant metadata that will then be filtered against user-criterion throughout the rest of the pipeline. The annotation stage is the only mandatory stage of the pipeline, and a great portion of filtering occurs at these stages too, with up to 90\% of true negatives being discarded.

As a result of the large demand placed upon the modules at this stage, they were written in the C++ in order to reduce time and memory constraints on low-end platforms. The stage is split into three modules (in order of processing): 

\begin{itemize}

\elem{Adding Genes}{(via \textit{GenePender}) Appends a gene-context to the variants under a user-configured level of detail at the gene/intergenic junction or the exon/intron/splice/UTR sub-divisions, including isoforms. Regulatory variants further up or downstream of UTR can be specified by defining custom margins of enclosement, and wholly intergenic regions are discarded by default (though can be kept upon user preference). 
}
\elem{Adding Function}{(via \textit{FuncAnnot}) Applies functional annotation upon the variants processed in the previous step; performing a cDNA lookup of where a variant falls within the coding portions of the gene in order to predict the type of mutation (missense, synonymous, or non-synonymous) at the codon and subsequent amino-acid level. Anti-sense encoded genes are handled accordingly, and for insertion/deletion (indels) variants the module performs the required addition/subtractions across a consistent reading frame to discern the mutation.
}
\elem{Adding Zygosity}{(via \textit{AddZygo}) Addresses a confidence issue in with pre-processed variants, where heterozygosity and homozygosity would be assigned based on post-quality filtering metrics. This module sets zygosity by nucleotide base-count alone, and determines HET/HOM assignments based upon a user-set frequency threshold (default $<0.65$).
}
\end{itemize}

%\textit{SNVIndel} simply determines whether a variant is Single Nucleotide Polymorhpism (SNV) or an Insertion-Deletion (InDel) compared to the reference genome

Once fully annotated, the resultant output VCFs are ready to be processed by the filtration modules.



\subsubsection{Filtration Modules}

The filtration modules consist of a series of Python (v2.7) scripts designed to parse these fields with the aim of minimizing the need for any mapping or additional pass-throughs.

A variant line in a VCF file describes six mandatory fields grouped into three distinct categories (in order of filtration complexity):

\begin{enumerate}
\belem{Variant Properties}{(CHROM) chromosome number, (POS) physical base-pair position, (REF) reference allele, and (ALT) alternate allele(s). These are processed by the following filtration modules:
\vspace{-5pt}
\begin{itemize}
	\elem{Physical Location Filter}{Parses the first two columns only; chromosome and physical base-pair position. A locus set is provided by the user and all variants that exist inclusively within are kept in the output.}
	\elem{Novel Variant Filter}{Parses the third column only describing variant identifier, and where not present (represented as '.') to keep the variant.}
	\end{itemize}
}

\belem{Variant Metadata}{(INFO) variant call information consisting of various call related properties summarizing the FASTA strand pileup it bisects.

	The INFO field consists of variant call report information which only alludes to the quality of the sample data, but not to the sample data itself, enabling for fast single-pass processing.

	\begin{itemize}
	\elem{Read Depth Filter}{Discards any variants falling below a user-set limit upon the number of FASTA reads aligned at that position.}
	\elem{Call Quality Filter}{The variant caller often assigns its own scoring nomenclature (non-transparent, often related to read-depth) which is processed or ignored at this step.}
	\elem{Mutation Type Filter}{Makes use of annotations acquired at the \textit{Adding Function} stage in order to filter single variants based upon user-set requirements of including any (multiple) of missense, nonsense, and synonymous mutation types.}
	\elem{Same Gene Filter}{Depending on the level of domain specificity (gene/exon), maps out all domains common across all input VCF files and produces an output set of VCFs that solely include variants falling within those domains only.}
	\elem{Same Variant Filter}{As previous, but under the more stringent requirement that all output variants match the same position.}

	\end{itemize}
}

\belem{Sample Data}{(FORMAT) Sample format field denoting the format which all subsequent sample data conform to.

	The sample data and format field cannot exist without the other, and it is required that the modules in this category process the format field before scanning the data. 

	\begin{itemize}
	\elem{Alternate Allele Frequency}{Scans the sample data in order ascertain the absolute frequencies of the alternate allele(s) in the population, removing variants with frequencies exceeding user-defined upper/lower-bound thresholds.}
	\elem{Inheritance Filter}{Requires multiple VCF inputs. Performs trait penetrance modelling for differently affected individuals following sibling-sibling, and sibling-parent relations.}
	\end{itemize}
}
\end{enumerate}



\subsubsection{Inheritance Filtering}

For all detected parent-offspring trios, variants undergo context-based filtering depending on the penetrance-model specified:

\begin{itemize}

\belem{Autosomal Dominant}{The phenotype is caused by a single mutant autosomal allele, and affected individuals must have affected parents, mapping any \{HOM,HET\}$\mapsto$\{HET,HOM\} under complete penetrance. Under a \textit{de novo} context all common affected variants are filtered against unaffected controls, otherwise variant commonality is kept within sibling groups.}

\belem{Autosomal Recessive}{The phenotype is caused by a loss of function stemming from both copies of an autosomal gene, likely from the result of consanguineous breeding. Two paths of transmission are considered from parent$\mapsto$offspring depending on whether the affected offspring variant is compound-heterozygous (C-HET) or homozygous (HOM):

	\begin{itemize}
	\elem{HOM}{At least one parent maps HOM$\mapsto$HOM, or \{HET/HET\}$\mapsto$HOM if both parents are carriers.}
	\elem{C-HET}{Parents are assumed to be carriers for different singular HET variants across common genes, which compound in offspring as multiple HET variants within the same gene. Under a gene context, this is resolved via \{HET1/HET2\}$\mapsto$\{HET1+HET2\} mapping to produce a C-HET gene.}
	\end{itemize}

Siblings are then filtered for common variants existing within affecteds siblings only, discarding those that are homozygous in unaffected controls.
}

\belem{X-linked Dominant}{As with autosomal dominant but with the mutant allele on the X-chromosome.}

\belem{X-linked Recessive}{As with autosomal recessive but with mutations occurring on the X-chromosome. Males with a single mutant copy are hemizygous and are treated as homozygous, exempting them from compound heterozygosity checking.}
\end{itemize}

Mosaicism is treated as a special case, where allele frequencies are pre-calculated for each variant and then filtered against user-set thresholds conforming to expected mosaic frequency ranges (typically between 10-35\%).


% P 37, 85, 107
% file:///home/tetris/Downloads/HSAP-122.pdf


\subsubsection{Extended Annotation}

The last stage of pipeline constitutes a small subset of variants which have successfully passed through the main filtering stages and require finer analysis which is enabled by providing an even greater context to compare the variants. Additional annotation relates to the downstream effects of said variants such as structure, function, and expression.

\begin{itemize}
\elem{Isoform Context}{Translates gene isoforms into their RefSeq nomenclature counterparts.}
\elem{Protein Context}{Assigns protein annotation information from UniProt sources to assign information related protein domain.}
\elem{Gene Expression}{Organ and tissue-specific data from the Encode GNF Atlas2 database is provided along with expression ratios which can be further filtered against user-specified limits.}
\end{itemize}




