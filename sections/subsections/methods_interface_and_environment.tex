
\subsection{Application Suite and Interface}

The pipeline was originally developed in a headless linux shell environment with input validation being left to the user, and run 


\subsubsection{Web-front}

To necessitate the uptake of \app, a web interface was created to facilitate input validation and pipeline configuration process.

The file upload procedure is streamlined by means of a pedigree file which pre-specifies cases (affecteds) and controls (unaffecteds) as well as their relation to one another. Pedigree data is automatically parsed into a file upload utility where the user can drag and drop their VCF files into the appropriate bins for processing.

The interface extends to display configuration options for each annotation and filtration module whilst uploading occurs in the background. Modules are enabled by expanding check-boxes to display individual module parameters and thresholds that can be overriden by user criterion, examples of which can be shown in Fig~\ref{fig:webend}. 

A drop-down box of available penetrance model provides mutually-exclusive model-dependent options to better refine the analysis, such as parent or unaffected sibling-specific filtering. Additional annotation requirements are set (or skipped upon preference) and then the pipeline is run in tandem to the existing input session.

In the case of user-termination, re-upload is not neccesary for the same analysis as the process will reuse the temporary files from the last session and will not repeat the same work twice, resuming from where it left off.

Once complete, the pipeline self-terminates and produces an interactive report of the remaining variants primed for feature presentation/concealment to help pinpoint variants of interest such as those shown in Fig~\ref{fig:report}

\placeholderimage{fig:report}
{Report of potential causative variants with dynamic filtering options.}

The pipeline is spawned in a GNU \textit{screen} session in order to enable process control and resumeablility, and snapshots of a session in-process are repeatedly retrieved from the shell process to the web front-end via PHP scripts. Ongoing \app processes can be managed both from the web-interface, as well as from the shell provided in the live environment.

%Conc: Future work would be to provide file snapshots at each stage of processing for better progress reporting, automatic pedigree-creation to 

\placeholderimage{fig:webend}
{Web-interface running an analysis}


\subsubsection{Self-Contained Environment}

The full \app suite consists of the core \app processing back-end encapsulated by the web-interface to handle input validation, which is encapsulated once more by an Arch Linux live environment that handles and provides general file utilities.

Each of these three components exist as separable peripherals, but are optimal in the above configuration by providing 




 handles symbolic links to static data as well as manages general file operations whilst providing

exists in a wrapped once by the web-interface that abstracts input parameter 


The core \app exists as a headless shell processing environment made up of programs, scripts, static data, and inter-connecting helper functions. The web-interface is an additional 

The core \app exists as the separable product of the web-interface front-end and the processing back-end with its constituent programs, scripts, and static data. 